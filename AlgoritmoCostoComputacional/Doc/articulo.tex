%%%%%%%%%%%%%%%%%%%%%%%%%%%%%%%%%%%%%%%%%%%%%%%%%%%%%%%%%%%%%%%%%%%%%%%%%%%
%
% Plantilla para un artículo en LaTeX en español.
%
%%%%%%%%%%%%%%%%%%%%%%%%%%%%%%%%%%%%%%%%%%%%%%%%%%%%%%%%%%%%%%%%%%%%%%%%%%%

% Qué tipo de documento estamos por comenzar:
\documentclass[a4paper]{article}
% Esto es para que el LaTeX sepa que el texto está en español:
\usepackage[spanish]{babel}
\selectlanguage{spanish}
% Esto es para poder escribir acentos directamente:
\usepackage[utf8]{inputenc}
\usepackage[T1]{fontenc}
\usepackage{float}

\usepackage{enumerate} 

%% Asigna un tamaño a la hoja y los márgenes
\usepackage[a4paper,top=3cm,bottom=2cm,left=3cm,right=3cm,marginparwidth=1.75cm]{geometry}

%% Paquetes de la AMS
\usepackage{amsmath, amsthm, amsfonts}
%% Para añadir archivos con extensión pdf, jpg, png or tif
\usepackage{graphicx}
\usepackage[colorinlistoftodos]{todonotes}
\usepackage[colorlinks=true, allcolors=blue]{hyperref}

%% Primero escribimos el título
\title{Analisis de algoritmos de ordenamiento }
\author{Bejar Merma Ángel Andrés\\
  \small Universidad Nacional de San Agustin\\
  \small andresbjar97@gmail.com\\
  \small Ciudad de Arequipa
  \date{}
}

%% Después del "preámbulo", podemos empezar el documento

\begin{document}
%% Hay que decirle que incluya el título en el documento
\maketitle

%% Aquí podemos añadir un resumen del trabajo (o del artículo en su caso) 
\begin{abstract}
El trabajo consiste en un análisis del rendimiento de un conjunto de algoritmos en función de su tiempo de ejecución con diferentes conjuntos de datos tanto para c++ como para Python.
Se  prueban, comparan y concluye qué algoritmo es mejor para conjuntos de datos pequeños, promedio, veremos el peor caso, caso promedio y mejor.

\end{abstract}

%% Iniciamos "secciones" que servirán como subtítulos
%% Nota que hay otra manera de añadir acentos
\section{Introducci\'on}

La ordenación es una de las cuestiones importantes  en computación  el problema de ordenación es importante porque de eso depende otros algoritmos los que los hacen mas o menos eficientes uno de de estos algoritmos es el algoritmo de búsqueda .

El concepto clave  es algoritmo ,un algoritmo se puede definir como una  secuencia de  pasos   bien definidos para resolver un problemas . El algoritmo toma una entrada y proporciona una salida.\cite{5376871}
Se han propuesto varios algoritmos de ordenación con diferentes restricciones, p. Ej. número de iteraciones (bucle interno, bucle externo), complejidad y problema de consumo de CPU.












\section{Marco Teórico}

Los algoritmos de ordenamiento se dividen en dos categorías bien diferenciadas:el Bubble Sort, Insertion Sort ,y Selection Sort en la categoría de O(n2), mientras que Quick Sort  y Merge Sort  
O(n log n).La descripción de cada uno de ellos a continuación.

\begin{enumerate}[a)]

\item \textbf{Bubble sort:}
El algoritmo de ordenacion básico es la clasificación de burbujas. Compara dos elementos adyacentes y realiza una operación de intercambio si se encuentra un pedido incorrecto con pasos repetidos. Esto también se denomina algoritmo de ordenación  por comparación.Una de sus ventajas es sus facil  implementacion 

\item \textbf{Heap sort:}

\item \textbf{Insertion sort}

\item \textbf{ Shell sort}

\item \textbf{Merge sort}

Este es un algoritmo de divide y conquista, con la ventaja  de fusionar listas con nuevas listas ordenadas. La complejidad del peor caso de la ordenación por fusión es O (nlog n), ya que podría usarse para conjuntos de datos grandes y peores. 
La ordenación por fusión utiliza los siguientes tres pasos \cite{merge} 

\begin{enumerate}[1)]
\item \textbf{Divide:}Si el tamaño de la matriz es mayor que 1, divídalo en dos subarreglos iguales de la mitad del tamaño.
\item \textbf{Conquista:} ordenar ambas subarreglos por recursividad
\item \textbf{Fusiona:} Combine ambas arreglos ordenadas en uno de tamaño original. Esto le dará un arreglo ordenada completo.


\end{enumerate}



 La ordenación por fusión es más adecuada para casos grandes y peores, pero usa más memoria en comparación con otros algoritmos de división y conquista. El algoritmo de clasificación de fusión se describe a continuación

\item \textbf{Quick sort}


\end{enumerate}


3.1 Clasificación de burbujas: el algoritmo de clasificación básico es la clasificación de burbujas. Compara dos elementos adyacentes y realiza una operación de intercambio si se encuentra un pedido incorrecto con pasos repetidos. Esto también se denomina algoritmo de clasificación por comparación [7]. El tipo de burbuja original lo hace 


\section{Metodología}

Lo que se hizo fue probar los algoritmos de ordenamiento  en Python 
utilizando Google colab y el entorno de ejecución GPU. La maquina que nos asigno google fue una Tesla k80 se intento reiniciar el entono para obtener una tesla t 40 , pero fue inútil.

Para la pruebas en c++
La computadora  que utilizo es una intel i5 de quinta generación  con 2 cores ,8 gb  de memoria ram 

Los algoritmos escogidos son los siguientes:

\begin{itemize}


\item  Bubble sort

\item Heap sort

\item Insertion sort

\item Selection sort

\item Shell sort

\item  Merge sort

\item Quick sort

como datos de entrada usaremos libreoffice para graficar los resultados 

\end{itemize}








\section{Resultados}


\begin{figure}[H]%primero aqui sino arriba sino abajo
\centering
\includegraphics[width=10cm]{imagenes/arq2.png}
\caption{}
\end{figure}

\begin{figure}[H]%primero aqui sino arriba sino abajo
\centering
\includegraphics[width=7cm]{imagenes/arquitectura2.png}
\caption{}
\end{figure}


\begin{figure}[H]%primero aqui sino arriba sino abajo
\centering
\includegraphics[width=14cm]{imagenes/cap.png}
\caption{}
\end{figure}

\begin{figure}[H]%primero aqui sino arriba sino abajo
\centering
\includegraphics[width=13cm]{imagenes/cpus.png}
\caption{}
\end{figure}


cuaderno de trabajo
\cite{colab}


\section{Conclusiones}


\section{Discusión}



Documentar sus hallazgos en un formato de artículo de investigación (formato de libre elección), considerando
aspectos del marco teórico (estado del arte), metodología, resultados, conclusiones, discusión y bibliografía \cite{fager}.

\bibliographystyle{plain}
\bibliography{referencias}




\end{document}
